\newpage


\section{Szkielet rozdziału czwartego - }

\subsection{Opis algorytmu}

W ramach algorytmu podziału siatki, zakładając m node'ów, każdy zawierający k rdzeni, wyróżniam dwa następujące etapy:

\begin{enumerate}
    \item Podział siatki na $m \cdot k$ obszarów.
    \item Podział siatki podzielonej na $m \cdot k$ obszarów na m obszarów.
\end{enumerate}

\subsubsection{Podział siatki na $m \cdot k$ obszarów}
W ramach tego etapu wyróżniam następujące podetapy:
        {\begin{enumerate}
             \item {mapowanie wejściowego pliku graficznego przedstawiającego początkową siatkę na graf,}
             \item {zmniejszanie grafu algorytmem LAM \cite{weighted_maching} do liczby wierzchołków równej liczbie partycji,
                 na które chcemy podzielić wejściową siatkę,}
             \item {przypisanie numerów partycji do wierzchołków w zmniejszonym grafie,}
             \item {stopniowe przywracanie grafu z jednoczesnym wyrównaniem krawędzi \cite{10.1007/3-540-44842-X_6} oraz balansowaniem pól obszarów (mniejsze
             obszary powiększają się kosztem większych),}
             \item {usunięcie szumów.}
\end{enumerate}}

\subsubsection{Podział siatki na $m$ obszarów}
W ramach tego etapu wyróżniam następujące podetapy:
\begin{enumerate}
    \item {zmniejszanie grafu algorytmem LAM \cite{weighted_maching} do liczby wierzchołków równej liczbie partycji,
        na które chcemy podzielić wejściową siatkę,}
    \item {przypisanie numerów partycji do wierzchołków w zmniejszonym grafie,}
    \item {udoskonalenie podziału.}
\end{enumerate}

\newpage
\section{Podział siatki na $m \cdot k$ obszarów - dokładny opis}

\subsection{Mapowanie obrazka na graf}
\begin{wrapfigure}{r}{0.5\textwidth}
    \vspace{-4mm}
    \includegraphics[width=\linewidth]{images/grid1}
    \caption{Obrazek, który reprezentuje strukturę siatki do podziału. Żółte obszary to obszary niepodzielne, czerwone to
    te wyłączone z obliczeń}
    \label{im:input}
\end{wrapfigure}
Stosunkowo krótki akapit o tym jak wygląda mapowanie.
Obrazki powinny być w formacie png lub jpg. Ten etap polega na przeiterowaniu po siatce w celu stworzenia grafu, który
ją odwzorowuje. Jest to części algorytmu, której nie było w żadnym z paperów, więc została ona w pełni stworzona przeze mnie.
Celem tej częsci algorytmu było stworzenie szybkiej metody tworzenia kolejnych scenariuszy testowych.