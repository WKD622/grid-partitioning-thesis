Rezultaty obliczeń były otrzymywane w podobny sposób jak w artykule \cite{1364754}.
Algorytm wywoływany był $100$ razy, a na końcu wybierany był najlepszy podział.
Jednak w artykule \cite{1364754} wybranie najlepszego wyniku odbywało się poprzez liczenie długości granic między obszarami,
a w moim przypadku były brane pod uwagę jeszcze inne kryteria.
Autorzy \cite{1364754} mogli w ten sposób zaprojektować wybieranie najlepszego podziału, ponieważ nie występowały u nich
obszary niepodzielne, przez które znacznie trudniej zdefiniować najlepszy podział.
W implementacji bez obszarów niepodzielnym, niemal pewne jest otrzymanie obszarów o równych polach.
Jeśli założymy, że nasza implementacja zawsze będzie w stanie zwrócić podział równy pod względem wielkości pól -
co da się zagwarantować na siatce bez obszarów niepodzielnych, ponieważ tam możliwości balansowania
pól są praktycznie nieograniczone - to pozostaje nam minimalizować sumaryczną długość granic.
Obszary niepodzielne sprawiają, że często pola partycji nie mogą być równe, a wybór najlepszego rezultatu wedle
kryterium najmniejszej długości granic może
prowadzić do wyłonienia podziału o dużej nierówności pod względem wielkości pól partycji, co pokażę na dalszych przykładach.
Kryterium wyłaniania najlepszego podziału było więc wybierane na podstawie wielkości i ułożenia obszarów niepodzielnych.
Jeśli znacznie utrudniały lub uniemożliwiały one wyrównanie powierzchni pól, a także poprzez swoje ułożenie powodowały wydłużenie długości granic,
wybierałem kryterium najmniejszej różnicy w wielkości pól, ponieważ był to ważniejszy parametr oraz
ponieważ dla tego typu przykładów algorytm często dawał obszary o stosunkowo dużych różnicach w wielkości pól.
Kiedy dowolność ułożenia partycji była duża z racji na małe obszary niepodzielne, kryterium była najmniejsza długość granic.
W przypadku siatek, dla których wybór kryterium nie jest oczywisty można wybrać najlepszy podział osobno wedle każdego kryterium,
a następnie podjąć decyzję na podstawie ich parameterów.
Kryterium najmniejszej różnicy w wielkości pól może być liczone za pomocą odchylenia standardowego bądź w mniej
skomplikowanych przypadkach
za pomocą różnicy w wielkości pola między największą i najmniejszą partycją.

Dla wszystkich wyników prezentowanych w rozdziale $5.1$ rysunek przedstawiający siatkę wejściową rysowany jest po lewej stronie.
Kolorem żółtym zaznaczone są obszary niepodzielne, natomiast kolorem czerwonym obszary wyłączone z obliczeń.
Odchylenie standardowe liczone jest dla procentowych udziałów wielkości partycji w całkowitej wielkości pola do podziału.
Dla wyników prezentowanych w rozdziale $5.2$ pierwszy rysunek od lewej prezentuje siatkę podzieloną na $m \cdot k$ obszarów.
Następny rysunek przedstawia jej partycjonowanie na $m$ obszarów po $k$ podobszarów każdy.
\vspace{-5mm}