Rezultaty obliczeń były otrzymywane w taki sam sposób jak w artykule \cite{1364754}.
Algorytm wywoływany był $100$ razu, na końcu wybierany był najlepszy podział.
O ile w artykule \cite{1364754} wybranie najlepszego wyniku odbywało się poprzez liczenie długości granic między obszarami,
tak w moim wypadku nie było to oczywiste.
Autorzy \cite{1364754} mogli tak zaprojektować wybieranie najlepszego podziału, ponieważ nie występowały u nich
obszary niepodzielne, przez które znacznie ciężej zdefiniować najlepszy podział.
W implementacji bez obszarów niepodzielnym to, co jest niezmienne to równe pola obszarów.
Jeśli założymy, że nasza implementacja zawsze będzie w stanie zwrócić podział równy pod względem pól,
co da się zagwarantować na siatce bez obszarów niepodzielnych, to pozostaje nam minimalizować sumaryczną długość granic.
Obszary niepodzielne sprawiają, że często pola partycji nie mogą być równe, a minimalizacja długości granic może
prowadzić do dużych nierówności w kwestii pól partycji co pokażę na dalszych przykładach.
Kryterium wyłaniania najlepszego podziału było więc wybierane często wedle mojej, subiektywnej oceny.
Przykładowo, kiedy obszary niepodzielne w znaczny sposób utrudniały podział, były to maksymalnie równe pola,
kiedy dowolność ułożenia była duża z racji na małe obszary niepodzielne, kryterium była długość granic.
Kryterium równych pól może być liczone za pomocą odchylenia standardowego bądź w mniej skomplikowanych przypadkach
za pomocą różnicy między największą i najmniejszą partycją.