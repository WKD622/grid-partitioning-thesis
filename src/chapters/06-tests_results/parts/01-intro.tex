Rezultaty obliczeń były otrzymywane w ten sam sposób jak w artykule \cite{1364754}.
Algorytm wywoływany był $100$ razy, a na końcu wybierany był najlepszy podział.
O ile w artykule \cite{1364754} wybranie najlepszego wyniku odbywało się poprzez liczenie długości granic między obszarami,
tak w moim wypadku nie było to oczywiste.
Autorzy \cite{1364754} mogli tak zaprojektować wybieranie najlepszego podziału, ponieważ nie występowały u nich
obszary niepodzielne, przez które znacznie ciężej zdefiniować najlepszy podział.
W implementacji bez obszarów niepodzielnym to, co jest niezmienne to równe pola obszarów.
Jeśli założymy, że nasza implementacja zawsze będzie w stanie zwrócić podział równy pod względem pól,
co da się zagwarantować na siatce bez obszarów niepodzielnych, ponieważ tam możliwości balansowania
pól są praktycznie nieograniczone, to pozostaje nam minimalizować sumaryczną długość granic.
Obszary niepodzielne sprawiają, że często pola partycji nie mogą być równe, a minimalizacja długości granic może
prowadzić do dużych nierówności w kwestii pól partycji, co pokażę na dalszych przykładach.
Kryterium wyłaniania najlepszego podziału było więc wybierane na podstawie wielkości i ułożenia obszarów niepodzielnych.
Jeśli znacznie utrudniały lub uniemożliwiały one wyrównanie pól, a także poprzez swoje ułożenie powodowały wydłużenie długości granic,
wybierałem kryterium najmniejszej różnicy w wielkości pól, ponieważ był to ważniejszy parametr oraz
ponieważ dla tego typu przykładów algorytm często dawał obszary o stosunkowo dużych różnicach w wielkości pól.
Kiedy dowolność ułożenia była duża z racji na małe obszary niepodzielne, kryterium była najmniejsza długość granic.
W przypadku siatek, dla których wybór kryterium nie był oczywisty można wybrać podział dla każdego kryterium osobno,
a następnie podjąć decyzję na podstawie ich parameterów.
Kryterium najmniejszej różnicy w wielkości pól może być liczone za pomocą odchylenia standardowego bądź w mniej
skomplikowanych przypadkach
za pomocą różnicy w wielkości pola między największą i najmniejszą partycją.

Dla wszystkich wyników prezentowanych w rozdziale $5.1$ rysunek przedstawiający siatkę wejściową rysowany jest po lewej stronie.
Kolorem żółtym zaznaczone są obszary niepodzielne, natomiast kolorem czerwonym obszary wyłączone z obliczeń.
Odchylenie standardowe liczone jest dla procentowych udziałów wielkości partycji w całkowitej wielkości pola do podziału.
Dla wyników prezentowanych w rozdziale $5.2$ pierwszy rysunek od lewej prezentuje siatkę podzieloną na $m \cdot k$ obszarów.
Następny rysunek przedstawia jej partycjonowanie na $m$ obszarów po $k$ podobszarów każdy.
\vspace{-5mm}