\subsection{Wyniki dla podziału na $m$ obszarów}

Ten podrozdział opisuje wyniki podziału siatki podzielonej na $m \cdot k$ partycji na $m$ partycji, z których każda
zawiera $k$ podobszarów.
Dla tej części zakładam, że podział na $m \cdot k$ partycji jest na równe, bądź niemal równe części i podaję
wyniki tylko dla podziału na $m$ partycji po $k$ podobszarów każda.

Różnica w postacji wywołania algorytmu dla tej części polega na tym, że wykonywanych jest $100$ iteracji
partycjonowania na $m \cdot k$ partycji, a dla każ∂ego z tych podziałów wykonywanie jest $100$ iteracji poszukiwania
najlepszego podzielenia tych $m \cdot k$ partycji na $m$ partycji po $k$ podobszarów każda.
Jest to możliwe, ponieważ szukanie $m$ ma niski koszt obliczeniowy.
Ze wszystkich wywołań wybierane jest ten podział na $m$ partycji, który ma najkrótszą długość granic.
Długość granic pod rysunkiem podawana jest dla podziału na $m$ obszarów.

\vspace{4mm}

Rysunek \ref{result:m:1} pokazuje partycjonowanie dla $k$ i $m$ wynoszącego $4$.
$16$ obszarów dzielone jest na $4$ partycje po $4$ podobszarów każda.
Podział na $m$ obszarów nie wykazał żadnych obszarów rozproszonych - każda partycja jest jedną całością,
a granica między obszarami jest krótka.
Wszystkie z czterech partycji mają niemal równe rozmiary, to znaczy, że wielkości partycji dla podziału (a) są
niemal idealnie równe.
\begin{figure}[h]
\centering
\begin{subfigure}{.33\textwidth}
    \centering
    \fbox{\includegraphics[width=0.7\linewidth]{images/results/m/1/mk}}
    \caption[short]{}
\end{subfigure}%
\begin{subfigure}{.33\textwidth}
    \centering
    \fbox{\includegraphics[width=0.7\linewidth]{images/results/m/1/m}}
    \caption[short]{}
\end{subfigure}
\begin{subfigure}{.33\textwidth}
    \centering
    \includegraphics[width=0.9\linewidth]{images/results/m/1/results}
    \caption[short]{}
\end{subfigure}
\caption{Siatka $50$x$50$. $k$ i $m$ wynosi $4$.
Sumaryczna długość granic dla tego wyniku wynosi $146$.
Wybór najlepszego rezultatu wedle kryterium najmniejszej długości granic.}
\label{result:m:1}
\end{figure}

Rysunek \ref{result:m:2} pokazuje partycjonowanie dla $k$ i $m$ wynoszącego $7$.
$49$ obszarów dzielone jest na $7$ partycji po $7$ podobszarów każda.
Po raz pierwszy widoczne są obszary rozproszone, są jednak one pojedyncze i tylko dla partycji $1$.
Wszystkie z siedmiu partycji mają niemal równe rozmiary, to znaczy, że wielkości partycji dla podziału (a) są
niemal idealnie równe.

Rysunek \ref{result:m:3} pokazuje partycjonowanie dla $k$ i $m$ wynoszącego $10$.
$100$ obszarów dzielone jest na $10$ partycji po $10$ podobszarów każda.
Wraz z liczbą obszarów rośnie liczba partycji rozproszonych, nie jest ona jednak duża.
Jest to tylko kilka pojedynczych obszarów.
Podział można uznać za udany.

Można zaobserwować, że wraz z liczbą partycji rośnie liczba partycji rozproszonych, nie jest to jednak bardzo intensywne
zjawisko.
Eksperymenty wykazały, że w celu znalezienia najlepszego partycjonowania lepiej wykonać więcej
podziałów na $m \cdot k$ partycji ($100$ prób), a następnie
dla każdego z nich podobną liczbę partycjowań na $m$ partycji po $k$ obszarów ($100$ prób),
niż wykonać mniej podziałów na $m \cdot k$ partycji ($1$-$10$ prób) i dla każdego
z nich znacznie więcej partycjowań na $m$ partycji po $k$ obszarów ($1000$-$3000$ prób).
Druga opcja jest możliwa, ponieważ wyliczanie partycjowań na $m$ partycji po $k$ obszarów jest bardzo mało kosztowne
obliczeniowo, ale dla tej opcji występuje znacznie więcej obszarów rozproszonych i długość granic między partycjami
jest większa.


\begin{figure}[h]
\centering
\begin{subfigure}{.33\textwidth}
    \centering
    \fbox{\includegraphics[width=0.7\linewidth]{images/results/m/2/mk}}
    \caption[short]{}
\end{subfigure}%
\begin{subfigure}{.33\textwidth}
    \centering
    \fbox{\includegraphics[width=0.7\linewidth]{images/results/m/2/m}}
    \caption[short]{}
\end{subfigure}
\begin{subfigure}{.33\textwidth}
    \centering
    \includegraphics[width=0.9\linewidth]{images/results/m/2/results}
    \caption[short]{}
\end{subfigure}
\caption{Siatka $50$x$50$. $k$ i $m$ wynosi $7$.
Sumaryczna długość granic dla tego wyniku wynosi $365$.
Wybór najlepszego rezultatu wedle kryterium najmniejszej długości granic.}
\label{result:m:2}
\end{figure}


\begin{figure}[h]
\centering
\begin{subfigure}{.33\textwidth}
    \centering
    \fbox{\includegraphics[width=0.7\linewidth]{images/results/m/3/mk}}
    \caption[short]{}
\end{subfigure}%
\begin{subfigure}{.33\textwidth}
    \centering
    \fbox{\includegraphics[width=0.7\linewidth]{images/results/m/3/m}}
    \caption[short]{}
\end{subfigure}
\begin{subfigure}{.33\textwidth}
    \centering
    \includegraphics[width=0.9\linewidth]{images/results/m/3/results}
    \caption[short]{}
\end{subfigure}
\caption{Siatka $50$x$50$. $k$ i $m$ wynosi $10$.
Sumaryczna długość granic dla tego wyniku wynosi $509$.
Wybór najlepszego rezultatu wedle kryterium najmniejszej długości granic.}
\label{result:m:3}
\end{figure}
