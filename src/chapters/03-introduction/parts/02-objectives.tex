\subsection{Cel pracy}
Celem niniejszej pracy jest skonstruowanie algorytmu tworzącego podział siatki na partycje.
Podział będzie się odbywał dla węzłów homogenicznych, a więc takich gdzie każdy rdzeń ma identyczną moc obliczeniową.
Partycje muszą więc być równe pod względem wielkości pola.
Dodatkową cechą takiego partycjonowania jest zminimalizowanie długości granic między partycjami (ang. edge-cut),
tak by narzut komunikacyjny był jak najniższy.
Początkowa siatka powinna pozwalać na zdefiniowanie obszarów niepodzielnych oraz obszarów wyłączonych z obliczeń.
Obszar niepodzielny nie może znajdować się w więcej niż jednej partycji, natomiast
obszar wyłączony z obliczeń powinien byc uwzględniany podczas partycjonowania, tak by wszystkie węzły obliczeniowe
obsługiwały równe pod względem pola obszary siatki, na których ma szansę odbyć się symulacja.

W celu wykonania podziału siatka zostanie zamieniona na odpowiadający jej graf, który
następnie przy wykorzystaniu proponowanych przez literaturę oraz zmodyfikowanych przeze mnie algorytmów grafowych
zostanie poddany partycjonowaniu.
Partycjonowanie będzie się odbywało dwukrotnie.
Jeśli bowiem mamy $m$ węzłów obliczeniowych, każdy po $k$ rdzeni, to siatka najpierw zostanie podzielona na
$m \cdot k$ możliwie równych pod względem pola obszarów, a następnie $m \cdot k$ obszarów zostanie podzielone
na $m$ obszarów, każdy po $k$ podobszarów.
Tak działający algorytm został opisany w niniejszej pracy.