\subsection{Cel pracy}
Celem tej pracy jest skonstruowanie algorytmu tworzącego podział siatki na partycje.
Podział ten będzie się odbywał dla węzłów homogenicznych, a więc takich gdzie każdy rdzeń ma identyczną moc obliczeniową.
Partycje więc muszą być równe pod względem pola.
Dodatkową cechą takiego partycjonowania jest zminimalizowanie długości granic między partycjami (ang. edge-cut),
tak by narzut komunikacyjny był jak najniższy.
Początkowa siatka powinna pozwalać na zdefiniowanie obszarów niepodzielnych oraz obszarów wyłączonych z obliczeń.
Obszar niepodzielny nie może znajdować się w więcej niż jednej partycji, natomiast
obszar wyłączony z obliczeń nie powinien być brany pod uwagę podczas partycjonowania, tak by wszystkie węzły obliczeniowe
obsługiwały równe pod względem pola obszary siatki, na których ma szansę odbyć się symulacja.

W celu partycjonowania siatka zostanie zamieniona na odpowiadający jej graf.
Graf następnie, przy wykorzystaniu proponowanych przez literaturę oraz zmodyfikowanych przeze mnie algorytmów grafowych
zostanie poddany partycjonowaniu.
Partycjonowanie będzie się odbywało dwukrotnie.
Jeśli bowiem mamy $m$ węzłów obliczeniowych, każdy po $k$ rdzeni, to siatka najpierw zostanie podzielona na
$m \cdot k$ możliwie równych pod względem pola obszarów, następnie $m \cdot k$ partycji zostanie podzielone
na $m$ partycji, każda po $k$ podobszarów.
W ten sposób funkcjonujący algorytm został opisany w tej pracy.