W tym rozdziale zostanie szczegółowo opisany algorytm podziału siatki, który został zaimplementowany na potrzeby
niniejszej pracy.
W ramach tego algorytmu, zakładając $m$ node'ów, każdy zawierający $k$ rdzeni, wyróżniam dwa następujące etapy:

\begin{enumerate}
    \item Podział siatki na $m \cdot k$ obszarów.
    \item Podział siatki podzielonej na $m \cdot k$ obszarów na $m$ obszarów.
\end{enumerate}

W ramach etapu podział siatki na $m \cdot k$ obszarów wyróżniam następujące podetapy:
        {\begin{enumerate}
             \item {mapowanie wejściowego pliku graficznego przedstawiającego początkową siatkę na graf,}
             \item {zmniejszanie grafu algorytmem LAM \cite{weighted_maching} do liczby wierzchołków równej liczbie partycji,
                 na które chcemy podzielić wejściową siatkę,}
             \item {przypisanie numerów partycji do wierzchołków w zmniejszonym grafie,}
             \item {stopniowe przywracanie grafu z jednoczesnym wyrównaniem krawędzi \cite{10.1007/3-540-44842-X_6} oraz balansowaniem pól obszarów (mniejsze
             obszary powiększają się kosztem większych),}
             \item {usunięcie obszarów rozproszonych.}
\end{enumerate}}

W ramach etapu podziału siatki na $m$ obszarów wyróżniam następujące podetapy:
\begin{enumerate}
    \item {zmniejszanie grafu algorytmem LAM \cite{weighted_maching} do liczby wierzchołków równej liczbie partycji,
        na które chcemy podzielić wejściową siatkę,}
    \item {przypisanie numerów partycji do wierzchołków w zmniejszonym grafie,}
    \item {udoskonalenie podziału.}
\end{enumerate}