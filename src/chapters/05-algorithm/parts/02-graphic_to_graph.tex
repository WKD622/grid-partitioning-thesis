W ramach etapu podziału siatki na $m \cdot k$ obszarów wyróżniam konwersję pliku graficznego na graf, redukcję rozmiaru grafu,
podział zredukowanego grafu oraz przywrócenie grafu do początkowego rozmiaru wraz z wprowadzeniem optymalizacji podziału
(wyrównanie powierzchni pól i zmniejszenie długości granic pomiędzy partycjami).

\subsubsection{Konwersja pliku graficznego na graf i redukcja obszarów niepodzielnych}

Pierwszą częścią algorytmu jest tworzenie grafu na podstawie siatki, która dostarczana jest jako dana wejściowa.
Siatka jest w formie pliku graficznego.
Przykład takiego pliku przedstawiony jest na rysunku \ref{im:input}.
Jeden piksel na siatce reprezentuje jeden wierzchołek w grafie, na którym dokonywane będzie partycjonowanie.
Kolor piksela decyduje o tym, jakiego typu wierzchołkiem będzie dany piksel oraz do jakiego typu obszaru należy.
Wierzchołki, które są tego samego koloru oraz sąsiadują ze sobą, są przyporządkowywane do tego samego obszaru.
Żółte piksele interpretowane są jako wierzchołki obszarów niepodzielnych, natomiast czerwone piksele
jako wierzchołki obszarów wyłączonych z obliczeń, białe piksele to wierzchołki zwyczajne.
Obszary wyłączone z obliczeń mogą być używane do zaznaczenia fragmentów siatki, na których symulacja
nie będzie miała miejsca.
Mogą być to na przykład ściany budynku lub pomieszczenia, które są wyłączone z symulacji.
Obszary niepodzielne to obszary, które z jakiś powodów nie powinny być podzielone na partycje,
na przykład wejścia do budynków.
Przynależność wierzchołków do konkretnych grup wpływa na ich parametry w grafie, takie jak waga.
Może również spowodować, że nie pojawią się na grafie.
Ta część algorytmu, choć stosunkowo mało skomplikowana, jest kluczową dla niniejszej pracy.
W związku z tym, że autorzy artykułu \cite{1364754} nie zawarli informacji w jakiej formie dostarczany był
wejściowy graf, ta część algorytmu została zrealizowana w całości przeze mnie.
Efekt zamiany siatki na graf widoczny jest rysunku \ref{im:input2}.

\begin{figure}[h]
    \centering
    \fbox{\includegraphics[width=0.6\linewidth]{images/grid1}}
    \caption{Obrazek, który reprezentuje strukturę siatki do podziału. Żółte obszary to obszary niepodzielne, czerwone to
    te wyłączone z obliczeń, białe to zwykłe przez które przechodzić będą granice podziału.}
    \label{im:input}
\end{figure}

Algorytm iteruje po pikselach obrazka wejściowego i na podstawie ich koloru tworzy wierzchołki grafu oraz buduje zbiory
obszarów niepodzielnych oraz wyłączonych z obliczeń.
W zbiorach przechowywane są numery wierzchołków.
Na początku każda krawędź pomiędzy wierzchołkami, a także
wierzchołki zwykłe oraz wierzchołki obszarów niepodzielnych otrzymują wagę $1$.
Wierzchołki obszarów wyłączonych z obliczeń mogą mieć przyporządkowaną wagę $0$ lub nie być mapowane na wierzchołki w grafie,
tworząc puste przestrzenie między wierzchołkami.
Każdy wierzchołek w grafie ma parametr, który określa, jakim jest typem obszaru.
Kolejnym etapem jest redukcja obszarów niepodzielnych do pojedynczych wierzchołków.
Każdy obszar niepodzielny zamieniany jest na wierzchołek o wadze równej sumie wag wierzchołków tego obszaru.
Od teraz wierzchołki, które reprezentują obszary niepodzielne, nie są rozróżniane jako obszary niepodzielne -
stają się zwykłymi wierzchołkami.
Ten proces przedstawiony jest na rysunku \ref{im:indivisible}.

W dalszych częściach algorytmu następuje partycjonowanie oraz przemieszczanie wierzchołków między partycjami.
Wszystkie te operacje będą następowały na grafie z co najmniej zredukowanymi obszarami niepodzielnymi.
W ten sposób obszary niepodzielne są zawsze traktowane jako całość, a pojedyncze wierzchołki nigdy nie zostaną od nich odłączone.
Wierzchołki wyłączone z obliczeń mają wagę $0$, ponieważ algorytmy wykorzystywane później do partycjonowania grafu, operują
głównie na parametrze wagi wierzchołków.
Waga wierzchołka w zredukowanym grafie reprezentuje liczbę wierzchołków z grafu początkowego, co ma wpływ
na balansowanie pól obszarów.
Chcemy, aby partycje były równe pod względem liczby wierzchołków, na których będą odbywać się obliczenia.
Waga $0$ wierzchołka oznacza, że nie jest on brany pod uwagę w wyrównywaniu pól obszarów, nie ma znaczenia dla algorytmu i
może być dowolnie przemieszczany pomiędzy partycjami.

\vspace{8mm}

\begin{figure}[h]
\centering
\begin{subfigure}{.2\textwidth}
    \centering
    \fbox{\includegraphics[width=0.6\linewidth]{images/strange7}}
    \caption[short]{}
\end{subfigure}%
\begin{subfigure}{.38\textwidth}
    \centering
    \fbox{\includegraphics[width=0.8\linewidth]{images/rm/2}}
    \caption[short]{}
\end{subfigure}%
\begin{subfigure}{.38\textwidth}
    \centering
    \fbox{\includegraphics[width=0.8\linewidth]{images/rm/1}}
    \caption[short]{}
\end{subfigure}
\caption{Obrazek (a) przedstawia obrazek wejściowy w rzeczywistym rozmiarze. Obrazki (b) oraz (c) przedstawiają
powstały graf. Obrazek (b) przedstawia sposób podejścia do obszarów wyłączonych z obliczeń, gdzie nie tworzone są
odwzorowujące je wierzchołki, obrazek (c) pokazuje podejście gdzie w ich miejsce tworzone są wierzchołki z wagą $0$.
Dla czytelności wierzchołki zwykłe zostały oznaczone kolorem szarym. }
\label{im:input2}
\end{figure}

\begin{figure}[h]
\centering
\begin{subfigure}{.5\textwidth}
    \centering
    \fbox{\includegraphics[width=0.6\linewidth]{images/in/1}}
    \caption[short]{}
\end{subfigure}
\begin{subfigure}{.5\textwidth}
    \centering
    \fbox{\includegraphics[width=0.6\linewidth]{images/in/2}}
    \caption[short]{}
\end{subfigure}%
\begin{subfigure}{.5\textwidth}
    \centering
    \fbox{\includegraphics[width=0.6\linewidth]{images/in/3}}
    \caption[short]{}
\end{subfigure}
\begin{subfigure}{.5\textwidth}
    \centering
    \fbox{\includegraphics[width=0.6\linewidth]{images/in/4}}
    \caption[short]{}
\end{subfigure}%
\begin{subfigure}{.5\textwidth}
    \centering
    \fbox{\includegraphics[width=0.6\linewidth]{images/in/5}}
    \caption[short]{}
\end{subfigure}
\caption{Obrazek (a) to wejściowa siatka.
Obszary niepodzielne są żółte, natomiast wyłączone z obliczeń czerwone.
Obrazki (b) oraz (d) pokazują siatkę tuż po konwersji na graf na dwa sposoby.
Sposób (b) nadaje wierzchołkom na obszarach wyłączonych z obliczeń wagę $0$, natomiast sposób (d) usuwa wierzchołki obszarów
wyłączonych z obliczeń.
Na wierzchołkach zaznaczona jest waga.
Można zaobserwować wagę $1$ dla wierzchołków zwykłych i tych budujących obszary niepodzielne.
Na tym etapie wszystkie krawędzie mają wagę $1$. Obrazek (c) oraz (e) przedstawiają ten sam graf po redukcji obszarów
niepodzielnych, odpowiednio z kroku (b) oraz (d).
Wierzchołek, do którego został zredukowany obszar niepodzielny otrzymuje powiększoną wagę, równą sumie wag wierzchołków, które
redukuje. Każda krawędź, która zastąpiła dwie krawędzie, zyskuje sumę wag tych krawędzi równą $2$.}
\label{im:indivisible}
\end{figure}

\FloatBarrier


