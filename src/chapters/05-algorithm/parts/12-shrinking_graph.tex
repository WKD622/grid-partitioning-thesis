W związku z tym, że korzystamy z $m$ węzłów obliczeniowych typu homogenicznego, gdzie każdy węzeł posiada $k$ rdzeni,
siatkę podzieloną na $m \cdot k$ części należy podzielić na $m$ obszarów, każdy składający się
z $k$ podobszarów.
Bardzo ważną obserwacją jest to, że w przeciwieństwie do poprzedniej części partycjonowania grafu na $m \cdot k$ obszarów,
gdzie dopuszczalne były nierówności w wielkości pól, tutaj bardzo ważne jest, aby każdy obszar składał
się z takiej samej liczby podobszarów.
Jest to poważne utrudnienie tego problemu.

\begin{figure}[h]
\centering
\begin{subfigure}{.5\textwidth}
    \centering
    \fbox{\includegraphics[width=0.8\textwidth]{images/m/1}}
    \caption[short]{}
\end{subfigure}%
\begin{subfigure}{.5\textwidth}
    \centering
    \fbox{\includegraphics[width=0.8\textwidth]{images/m/2}}
    \caption[short]{}
\end{subfigure}
\caption{Obrazek (a) podział na $m \cdot k$ obszarów. Obrazek (b) przedstawia podział $m \cdot k$ obszarów na $m$ obszarów
po $k$ podobszarów. $m=4$ oraz $k=4$.}
\label{im:k_m}
\end{figure}

Ta część algorytmu jest realizowana przez algorytm weighted matching z dodatkiem części zachłannej.
Daną wejściową do algorytmu jest siatka z podziałem na $m \cdot k$ obszarów, jak na rysunku \ref{im:k_m}(a).
Najpierw wszystkie partycje redukowane są do pojedynczych wierzchołków.
Oznacza to, że mamy $m \cdot k$ wierzchołków.
Następnie graf składający się z $m \cdot k$ wierzchołków redukowany jest do $m$ wierzchołków
za pomocą algorytmu weighted matching, identycznego jak na rysunku \ref{code:lam}.
Po zredukowaniu do $m$ wierzchołków, do każdego wierzchołka przypisywana jest inna partycja.
Następnie graf jest odtwarzany do rozmiaru $m \cdot k$ z zachowaniem nowych partycji.
W tym momencie otrzymujemy podział $m \cdot k$ wierzchołków na $m$ obszarów.

Charakterystyczną cechą algorytmu weighted matching jest to, że od idealnie równych podziałów bardziej ceni sobie obszary
''zwarte'', czyli takie, które mają możliwie krótkie granice.
Jest to istotnym elementem jego charakterystyki, którego nie da się wyeliminować.
W związku z powyższym, większość wywołań algorytmu partycjonowania
nie kończy się podziałem $m$ obszarów po $k$ podobszarów każdy, natomiast podziałem bliskim temu podziałowi, bo choć LAM nie gwarantuje
podziałów idealnie równych, gwarantuje powstanie obszarów o zbliżonej liczbie wierzchołków.

Czasami jednak na tym etapie otrzymujemy już $m$ obszarów po $k$ podobszarów każdy.
Jeśli tak nie jest, to uruchamiana jest
zachłanna część algorytmu, która wyrównuje liczbę podobszarów pomiędzy obszarami.
Składa się ona z dwóch części.
Najpierw algorytm sprawdza jakie obszary leżą obok siebie i wyrównuje pola sąsiadów.
Przykładowo, jeśli leżą obok siebie dwa obszary $A$ oraz $B$, z czego $A$ zawiera $k-1$ podobszarów, natomiast $B$ $k+1$
podobszarów, to jeden z podobszarów graniczych przenoszony jest z $B$ do $A$ w celu wyrównania liczby
podobszarów do $k$ dla każdego z nich.
Jeśli $B$ zawierałby $k+3$ obszary, natomiast A $k-1$, to z $B$ przeniesione zostaną dwa obszary do $A$.
W ten sposób podobszary z dużych obszarów mają sposobność ''rozproszyć'' się po siatce.
W kolejnej turze nadmiarowe obszary z $A$ mają szansę zostać przeniesione gdzie indziej.
Ta część algorytmu wywoływana jest dopóki nie przestanie przynosić nowych zmian.
Ponieważ operujemy na wierzchołkach, które mają duże wagi, to już na tym etapie przesunięcia obszarów powodują często
tworzenie się obszarów rozproszonych.

Jeśli po tym etapie nadal nie mamy równych obszarów, następuje ostatni etap algorytmu.
Ten etap działa dopóki wszystkie obszary nie będą miały takiej samej liczby wierzchołków.
Wybierany jest zawsze obszar o największej i najmniejszej liczbie podobszarów.
Następnie z obszaru o większej liczbie podobszarów losowne są podobszary, które są przenoszone do
obszaru o mniejszej liczbie podobszarów, dopóki wybrana para nie będzie miała równego rozmiaru.

Zaletą tego algorytmu jest niski koszt obliczeniowy oraz prostota - im bardziej udany podział zwrócony przez algorytm
weighted matching, tym większa szansa na dobry podział końcowy.
Liczby wierzchołków, na których operuje algorytm, są bardzo małe, więc można wielokrotnie
wywoływać cały algorytm w poszukiwaniu najlepszego rezultatu.
Jego prostota oraz zachłanny charakter sprawiają jednak, że
jakość podziałów pod względem długości granic może nie być najwyższa - rysunek \ref{im:k_m2}.

\begin{figure}[h]
\centering
\begin{subfigure}{.5\textwidth}
    \centering
    \fbox{\includegraphics[width=0.8\textwidth]{images/m/3}}
    \caption[short]{}
\end{subfigure}%
\begin{subfigure}{.5\textwidth}
    \centering
    \fbox{\includegraphics[width=0.8\textwidth]{images/m/4}}
    \caption[short]{}
\end{subfigure}
\caption{Obrazek (a) podział na $m \cdot k$ obszarów. Obrazek (b) przedstawia podział $m \cdot k$ obszarów na $m$ obszarów
po $k$ podobszarów. $m=4$ oraz $k=4$.}
\label{im:k_m2}
\end{figure}