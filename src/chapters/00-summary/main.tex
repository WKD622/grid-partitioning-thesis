\section{Streszczenie pracy}
Partycjonowanie siatek jest bardzo ważnym podproblemem dla wielu dziedzin.
Zastosowanie ma między innymi dla sieci komputerowych, także dla obliczeń równoległych, gdzie zależy nam,
aby problem podzielić na równe części pomiędzy węzły.
Jednym z takich problemów są symulacje $2$D.
Dla symulacji definiujemy zasady jej odbywania oraz siatkę, na której ma się odbywać.
Celem takiej symulacji może być przykładowo zbadanie ruchu pieszego na mapie galerii handlowej, następnie wyodrębnienie miejsc
zatłoczonych, które wymagają poprawek w projekcie.
Dzieląc siatkę na równe partycje możemy podzielić obliczenia, które się niej odbywają, na węzły obliczeniowe.
Na potrzeby rozwiązania siatka zamieniana jest na graf.
Niniejsza praca proponuje algorytm partycjonowania grafów, który bazuje na metodzie Helpful-Sets.
Został on odtworzony oraz rozszerzony względem oryginalnego rozwiązania o uwzględnianie obszarów niepodzielnych
oraz wyłączonych z obliczeń.
Zaprezentowane są szczegóły implementacyjne algorytmu, a także wyniki jego działania.
Jest on także porównywany do innych rozwiązań.

\newline
\vspace{10mm}

Słowa kluczowe: Heuristic Graph Partitioning, Helpful-Sets, Grid partitioning.