\section{Streszczenie pracy}
Partycjonowanie siatek jest bardzo ważnym podproblemem dla wielu dziedzin.
Zastosowanie ma między innymi dla sieci komputerowych, także dla obliczeń równoległych, gdzie zależy nam,
aby problem podzielić na równe części pomiędzy węzły.
Jednym z takich problemów są symulacje $2$D.
Dla symulacji definiujemy zasady jej odbywania oraz siatkę, na której ma się odbywać.
Celem takiej symulacji może być przykładowo zbadanie ruchu pieszego na mapie galerii handlowej, następnie wyodrębnienie miejsc
zatłoczonych, które wymagają poprawek w projekcie.