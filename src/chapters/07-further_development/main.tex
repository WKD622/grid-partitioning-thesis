\newpage
\section{Podsumowanie i kierunek rozwoju}

W ramach niniejszej pracy została odtworzona oraz zmodyfikowana metoda partycjonowania grafów używana przez
bibliotekę Party \cite{1364754}.
Celem modyfikacji było partycjonowanie siatek dla efektywnego zrównoleglenia symulacji 2$D$.
Symulacje przeprowadzane są na $m$ węzłach, każdy zawierający $k$ rdzeni.
Siatka musi być podzielona na $m$ równych partycji po $k$ równych podobszarów każda.
Ważnym wymaganiem było dzielenie z zachowaniem możliwie krótkich granic między obszarami, ponieważ im dłuższe granice,
tym większy koszt komunikacji.

Algorytm został zrealizowany w dwóch etapach.
Pierwszy dzielił siatkę na $m \cdot k$ partycji.
Drugi dzielił siatkę podzieloną na $m \cdot k$ partycji na $m$ partycji, każda po $k$ równych podobszarów.
Nowym elementem było uwzględnienie przy partycjonowaniu obszarów niepodzielnych oraz wyłączonych
z obliczeń.

Pierwszy etap podziału na $m \cdot k$ partycji był najbardziej czasochłonnym elementem pracy.
Odtworzenie algorytmu oraz dostosowaniu go pod rozszerzone wymagania zakończyło się sukcesem.
Algorytm otrzymał wyniki bliskie bibliotekom uznawanym przez literaturę jako dające wyniki state-of-the-art.
Ponadto obsługuje on nieobsługiwane przez te biblioteki obszary niepodzielne oraz wyłączone z obliczeń.

Drugi etap został zrealizowany przy pomocy narzędzi stworzonych na potrzeby pierwszego, dlatego wyniki nie są
aż tak dobre.
Bazuje głównie na wielokrotnym powtarzaniu obliczeń, a następnie na wybieraniu najlepszych rezultatów.
Algorytm wybrany do tego etapu nie do końca nadaje się do dzielenia na partycje o idealnie równej liczbie podobszarów,
dlatego musiał zostać dopełniony algorytmem zachłannym.

Cel pracy został osiągnięty.
Jako dalsze kierunki rozwoju wskazałbym:
\begin{enumerate}
    \item {Poprawienie implementacji etapu pierwszego, w celu otrzymania wyników bliższych oryginalnej implementacji.
    Oznacza to poprawienie wyników dla długości granic, ale w szczególności poprawienie wydajności,
    ponieważ czasy wykonywania stanowiły największą różnice w stosunku do oryginalnej implementacji.}
    \item {Rozwinięcie drugiego etapu poprzez zaproponowanie innego algorytmu, który w swojej charakterystyce
    od samego początku został zaprojektowany pod kątem otrzymywania obszarów o równej liczbie wierzchołków lub też
    udoskonalenie aktualnego rozwiązania.}
\end{enumerate}