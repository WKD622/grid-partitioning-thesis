\documentclass[polish,12pt]{aghthesis}
% \documentclass[english,12pt]{aghthesis} dla pracy w jêzyku angielskim. Uwaga, w przypadku strony tytu³owej zmiana jêzyka dotyczy tylko kolejno¶ci wersji jêzykowych tytu³u pracy. 

% Szablon przystosowany jest do druku dwustronnego. 



\author{Jan Kowalski, Jan Malinowski\\ Wojciech Kowalski}

\titlePL{Spo³eczno¶ciowy system wspomagaj±cy zarz±dzanie odtwarzaniem muzyki w obiektach us³ugowych i u¿yteczno¶ci publicznej}
\titleEN{Sketches of the particular foundations of general theory of everything}

\fieldofstudy{Informatyka}

\supervisor{dr hab.\ in¿.\ Krzysztof Iksiñski, prof.\ nadzw.\ AGH}

\date{\the\year}


\begin{document}

\maketitle

\section{\SectionTitleProjectVision}
\label{sec:cel-wizja}
\emph{Charakterystyka problemu, motywacja projektu (w tym przegl±d
  istniej±cych rozwi±zañ prowadz±ca do uzasadnienia celu prac),
  wizja produktu i analiza zagro¿eñ.}  % niniejsza linijka to tylko komentarz, który nale¿y usun±æ

\section{\SectionTitleScope}
\label{sec:zakres-funkcjonalnosci}
\emph{Kontekst u¿ytkowania produktu (aktorzy, wspó³pracuj±ce systemy)
  oraz specyfikacja wymagañ funkcjonalnych i niefunkcjonalnych.}  % niniejsza linijka to tylko komentarz, który nale¿y usun±æ

\section{\SectionTitleRealizationAspects}
\label{sec:wybrane-aspekty-realizacji}
\emph{Przyjête za³o¿enia, struktura i zasada dzia³ania systemu,
  wykorzystane rozwi±zania technologiczne wraz z uzasadnieniem
  ich wyboru, istotne mechanizmy i zastosowane algorytmy.} % niniejsza linijka to tylko komentarz, który nale¿y usun±æ

\section{\SectionTitleWorkOrganization}
\label{sec:organizacja-pracy}
\emph{Struktura zespo³u (role poszczególnych osób), krótki opis i
  uzasadnienie przyjêtej metodyki i/lub kolejno¶ci prac, planowane i
  zrealizowane etapy prac ze wskazaniem udzia³u poszczególnych
  cz³onków zespo³u, wykorzystane praktyki i narzêdzia w zarz±dzaniu
  projektem.}  % niniejsza linijka to tylko komentarz, który nale¿y usun±æ

\section{\SectionTitleResults}
\label{sec:wyniki-projektu}
\emph{Wskazanie wyników projektu (co konkretnie uda³o siê uzyskaæ:
  oprogramowanie, dokumentacja, raporty z testów/wdro¿enia, itd.), prezentacja wyników
  i ocena ich u¿yteczno¶ci (jak zosta³o to zweryfikowane --- np.\ wnioski
  klienta/u¿ytkownika, zrealizowane testy wydajno¶ciowe, itd.),
  istniej±ce ograniczenia i propozycje dalszych prac.}  % niniejsza linijka to tylko komentarz, który nale¿y usun±æ

% o ile to mo¿liwe proszê uzywaæ odwo³añ \cite w konkretnych miejscach a nie \nocite

\nocite{artykul2011,ksiazka2011,narzedzie2011,projekt2011}

\bibliography{bibliografia}

\end{document}
