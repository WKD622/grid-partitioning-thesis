%##noBuild
\documentclass[polish,12pt]{input/aghthesis}

\usepackage[utf8]{inputenc}
\usepackage[T1]{fontenc}
\usepackage[MeX]{polski}

\usepackage{graphicx}
\usepackage{listings}
\usepackage{xcolor}
\usepackage{pdflscape}
\usepackage{lmodern}
\usepackage[hidelinks]{hyperref}
\usepackage[toc,page]{appendix}
\usepackage{calc}
\usepackage[nomessages]{fp}
\usepackage{dirtree}
\usepackage{amsmath}
\usepackage{tcolorbox}
\usepackage{url}
\usepackage{tikz}
\usepackage{wrapfig}
\usepackage{caption}
\usepackage{lipsum}
\usepackage{subcaption}
\usepackage{float}
\usepackage{soul}
\usepackage{placeins}

\usetikzlibrary{matrix,calc}

\setlength{\parindent}{0pt}

\definecolor{codegreen}{rgb}{0,0.6,0}
\definecolor{codegray}{rgb}{0.5,0.5,0.5}
\definecolor{codepurple}{rgb}{0.58,0,0.82}
\definecolor{pythonbackcolour}{rgb}{1,1,0.94}
\definecolor{consolebackcolour}{rgb}{0.32,0.32,0.35}
\definecolor{lightblue}{RGB}{219, 255, 250}
\definecolor{mygray}{RGB}{250, 250, 250}
\definecolor{purple}{RGB}{249, 255, 229}

\lstdefinestyle{pseudocodestyle}{
    frame=single,
    backgroundcolor=\color{mygray},
    commentstyle=\color{codegreen},
    keywordstyle=\color{magenta},
    numberstyle=\tiny\color{codegray},
    stringstyle=\color{codepurple},
    basicstyle=\ttfamily\footnotesize,
    breakatwhitespace=false,
    breaklines=true,
    captionpos=b,
    keepspaces=true,
    numbers=left,                    
    numbersep=5pt,                  
    showspaces=false,                
    showstringspaces=false,
    showtabs=false,                  
    tabsize=2,
    showlines=true,
    aboveskip=2em,
    belowskip=3em,
}

\lstdefinestyle{consoleerrorstyle}{
	rulecolor=\color{black},
	frame=single,
    backgroundcolor=\color{mygray},   
    commentstyle=\color{codegreen}\textit,
    keywordstyle=\color{red},
    numberstyle=\tiny\color{codegray},
    stringstyle=\color{red},
    basicstyle=\ttfamily\footnotesize\color{red},
    breakatwhitespace=false,         
    breaklines=true,                 
    captionpos=b,                     
    keepspaces=true,                               
    numbersep=5pt,                  
    showspaces=false,                
    showstringspaces=false,
    showtabs=false,                  
    tabsize=2,
    showlines=true,
    aboveskip=2em,
    belowskip=3em,
}
 
\lstdefinestyle{consolestyle}{
	frame=single,
    backgroundcolor=\color{mygray},   
    commentstyle=\color{codegreen}\textit,
    keywordstyle=\color{black},
    numberstyle=\tiny\color{codegray},
    stringstyle=\color{codepurple},
    basicstyle=\ttfamily\footnotesize,
    breakatwhitespace=false,         
    breaklines=true,                 
    captionpos=b,                     
    keepspaces=true,                               
    numbersep=5pt,                  
    showspaces=false,                
    showstringspaces=false,
    showtabs=false,                  
    tabsize=2,
    showlines=true,
    aboveskip=2em,
    belowskip=3em,
}

\lstdefinestyle{importstyle}{
	frame=single,
    backgroundcolor=\color{pythonbackcolour},   
    commentstyle=\color{codegreen},
    keywordstyle=\color{magenta},
    numberstyle=\tiny\color{codegray},
    stringstyle=\color{codepurple},
    basicstyle=\ttfamily\footnotesize,
    breakatwhitespace=false,         
    breaklines=true,                 
    captionpos=b,                     
    keepspaces=true,                                   
    numbersep=5pt,
    showspaces=false,
    showstringspaces=false,
    showtabs=false,
    tabsize=2,
    showlines=true,
    aboveskip=1em,
    belowskip=3em,
}


\lstnewenvironment{pseudocode}
{\lstset{style=pseudocodestyle, mathescape=true, escapeinside=@@}}
{}

\lstnewenvironment{python}
{\lstset{language = python, style=pythonstyle}}
{}

\lstnewenvironment{import}
{\lstset{language = python, style=importstyle}}
{}

\lstnewenvironment{bash}
  {\lstset{language=bash, style=consolestyle}}
  {}
  
\lstnewenvironment{consolerror}
  {\lstset{language=bash, style=consoleerrorstyle}}
  {}
  
\newcommand{\myspace}{\vspace{6mm}}

\newcommand{\TODO}[1]{
\myspace
{\huge TODO}
[ #1 ]
\myspace
}


\author{Jakub Ziarko}

\titlePL{Optymalizacja podziału dyskretnego modelu symulacyjnego na potrzeby symulacji równoległej}
\titleEN{Optimization of the discrete simulation model division for parallel simulation}

\fieldofstudy{Informatyka}
\type{Stacjonarne}
\supervisor{dr hab. inż. Wojciech Turek}

\date{\the\year}

\newtheorem{definition}{Definition}